\chapter{Electrical Equipment}

The Schaublin 13 Universal Miller is always delivered with complete
electrical equipment (motors, switches, motor protection switches
and wiring) ready for connecting to the mains. Before connecting
the machine up to the mains, make sure that the voltage specified
on the machine data plate tallies with that of the mains supply.
For wiring diagram, see page 14.

\section*{Connecting Up}

The machine is normally supplied with three-phase motors. The mains
connection terminals RST are located in the terminal box 13, this
being in the base. The machine is earthed via the yellow terminal
in the terminal box. When the current is switched on, make sure that
the motors are running in the correct direction, the procedure being
as follows:-

Place the screw 14 in the centre of that of the 2 milling cutters
outlined on the plate that corresponds to the cutter fitted to the
milling spindle, and turn the control lever 15 in the direction now
possible. The milling spindle must now rotate in the direction
indicated on the plate. (See page 11).

\section*{Description of the Equipment}

The Miller is equipped with three motors; one for the milling
spindle drive and table feeds, one for the high-speed table feed,
and one for the coolant pump.

\subsection*{Spindle and Table-feed Motor 16}

Type Oerlikon N84-2-4, three-phase, short-circuited rotor, power
2 HP, speed 1410 r.p.m. Connection box on right as viewed from end
of shaft.

\subsection*{High-speed Feed Motor 17}

Type Schindler KDWF03, power 0,3 HP, speed 2800 r.p.m. 3 x 3 mm.
key on end of shaft.

\subsection*{Coolant Pump Motor 18}

Vertical type, power 0,1 HP, speed 2800 r.p.m.
The spindle and pump motors are controlled and protected by the
"CHC" switching contactors 19, type VTp 15, mounted in the machine
frame. In the event of the motors being overloaded for a protracted
period, these contactors automatically switch off. The motors are
restarted by pressing push-buttons 20 ans 21 (See page 17).

Switch 22 is used for switching on and reversing the milling
spindle motor, lever 15 being moved to left or right according
to the position of screw 14.

Switch 23 enables the high-speed feed motor to be switched on
by means of a pedal 24 provided in front of the base.

This motor is protected from overloading by a friction clutch.

The coolant pump motor is switched on by means of the push-
button 21 of the contactor located underneath the plate showing
a cock.

Directions concerning the motors are attached to these present
instructions.

\section*{Starting Up}

When all instructions regarding erection, cleaning, lubrication
and electrical equipment have been carried out, start the machine
up and let it run idle for several hours. Start with a low speed.
to enable the bearings and transmission members to warm up
normally, then gradually increase to maximum speed. Check the
proper functioning of each part.
