\chapter{Service Instructions for Alternating Current Motors with Short-Circuit Rotor}

\begin{figure}[ht]
    \centering
    \includegraphics[width=0.1\linewidth]{images/page_63_manufacturer_logo}
    \label{fig:schindler_manufacturer_logo}
\end{figure}

\begin{center}
    \textbf{\small SCHINDLER \& CIE. S.A. EBIKON-LUCERNE, SWITZERLAND}
\end{center}


\section*{1. Installation}

The motor will be placed as much as possible in a place free of dust and humidity;
it should be fixed, plumb, on a metal or concrete base (or wood for small motors).

The motor and device housings must be grounded.

The motor connection will be made according to the connection diagram attached to each machine.
The motor must be protected according to the intensity indicated on the plate and according to the starting conditions.

For installations with a double-width pulley, make sure that, during loaded operation, the belt is on the motor side.

As a transmission element, glued, leather, or balata belts will be used. The use of staples should be avoided.
Axial shocks of the rotor are an indication of a deformed belt. To prevent bearing heating, the belt should not be too tight.
If the pulley needs to be replaced or if a second pulley is key-seated on the other end of the shaft,
make sure that these pulleys are well balanced.
Dismounting or keying of the pulleys must be done carefully to avoid damaging the bearings.

\section*{2. Commissioning}

Before commissioning, the sleeve bearings must be carefully cleaned with petroleum.
The latter will be introduced through the bearing cover and will flow out through the drain hole whose cap will have been unscrewed.
This operation will be repeated until the petroleum flows clear through the drain hole.
