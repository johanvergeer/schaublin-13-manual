\section*{4. Maintenance}

After removing the outer cover of the bearing and loosening the screws fixing the bearing flanges to the frame,
the flanges can be moved, leaving the bearings easily accessible and fixed on the shaft.
Remove the old grease from the covers and shields by washing them with petroleum.
Then, fill the covers and bearings about three-quarters full with new grease and reassemble everything,
putting each piece back in its original position before disassembly.
It is essential to avoid the introduction of water or dust into the bearings; the seals should be in good condition,
and the screws tightened. Only use high-quality grease, free of acid, retaining its lubricating qualities at high temperatures
as well as in cold conditions.
The replacement of pads or ball bearings is determined by the air gap (clearance between the stator and rotor),
which should not be less than 0.1-0.2 mm. This gap can be checked using a gauge that we can provide upon request.

All parts are manufactured according to gauges and are, therefore, interchangeable.
In case of ordering any spare parts, it is sufficient to indicate the manufacturing number marked on the motor plate.

\section*{4. Faults and Their Causes}

\subsection*{1. Bearing Overheating}
\begin{itemize}
    \item Defective assembly when the motor is directly coupled with a machine, and the two shaft ends are not aligned.
    \item Insufficient cleaning of bearings before commissioning.
    \item Grease rings not rotating; the oil is too thick or excessive in the bearings.
    \item Poor-quality, impure, or insufficient oil.
    \item Too much or too little grease or poor-quality grease in the bearings.
    \item Belt too tight, or in the case of a double-width pulley, the belt under load is too far from the bearing.
\end{itemize}

\subsection*{2. Oil Losses at the Bearings}
\begin{itemize}
    \item Excessive oil in the bearings.
    \item Oil channels are obstructed.
    \item The drain screw is not sufficiently tightened.
\end{itemize}

\subsection*{3. Abnormal Motor Noise}
\begin{itemize}
    \item Cut in a pipeline (defective fuse, poor contact, or lack of contact at the switch, disconnected connections on the terminal board).
    \item Short circuit between windings.
    \item Bearings misaligned due to wear.
\end{itemize}

\subsection*{4. Excessive Motor Heating}
\begin{itemize}
    \item Due to overload (excessive current).
    \item Insufficient motor cooling (surrounded by a partition).
    \item Voltage too high.
    \item Ventilation channels clogged.
    \item Short circuit in windings.
\end{itemize}

\subsection*{5. Motor Stops}
\begin{itemize}
    \item Due to overload.
    \item Too low voltage.
    \item Interruption in a pipeline (defective fuse, poor or no contact at the switch).
    \item Seizure of plain bearings or defective bearings.
\end{itemize}

\subsection*{6. Abnormal Number of Revolutions}
\begin{itemize}
    \item Due to overload.
    \item Interruption in a phase.
    \item Frequency deviations.
\end{itemize}

\begin{flushright}
    Fabrique d'Ascenseurs et de Moteurs Electriques\\
    \textbf{SCHINDLER \& CIE. S.A. EBIKON-LUCERNE}\\
    Phone (041) 6 21 21\\
    Phone (041) 6 31 31
\end{flushright}
