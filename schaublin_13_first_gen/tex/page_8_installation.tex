\chapter{Erection}

\section*{Handling}
On receipt of the machine, uns crew the top of case and remove
sides, unscrew the four fixing bolts and take out the machine.
Take care to recover any accessories which may be included among
the packing.
The machine weighs approximately 500 kilos (1100 lbs.). When
handling the machine with the aid of hoisting tackle, it should
be slung as shown in the drawing on page 8. The hook should be
wrapped with rags and the table is locked in its bottom position
(without the swarf tray).

\section*{Concrete Foundation}
The Schaublin 13 Miller is designed for erection on a concrete
foundation, the dimensions of the latter being as shown in the
foundation drawing on page 8. The depth of the foundation will
depend on the nature of the ground; the concreting should be done
on firm ground only.
The electric supply leads can be laid either above ground,
through the gap between the floor and the frame, or underground
along a channel leading to point 4. In this latter case it is
advisable to provide in the concrete foundation a steel pipe 2
with an inside diameter of 26 mm., 155 mm. long. The electric
supply cable must project about 24" above the floor.
The machine is secured to the concrete foundation by means of
4 stay-bolts 1, these latter being first inserted in the holes
provided in the concrete. With the aid of 4 flat steel strips 3,
placed under the feet of the machine, the latter is levelled
and vertically trued by means of a precision spirit-level.
When this has been done, the staybolts, iron strips and machine
feet are grouted in with cement, care being taken to keep the
machine correctly levelled and trued.
The holes provided in the machine base to accommodate the fixing
bolts are 14 mm. in diameter. The stay-bolts, nuts and strips
for fixing the machine are not supplied with the latter.
The machine must be accessible from every direction. (see page 6). % TODO: add page reference

\section*{Cleaning}
Only clean, chemically neutral and preferably white rags should
be used for degreasing and cleaning.
First remove the anti-rust grease with a dry scouring cloth,
and then wipe the surfaces over with a rag dipped in paraffin
and wrung out. The anti-rust grease has no lubricating proper-
ties whatever, and must be completely removed as its presence
may cause serious seizing, often weeks after the machine has
been started up. Care should be taken during cleaning operations
to ensure that no scratches are produced, especially on the
saddle and milling-spindle head guides.
Finally, all bright surfaces should be coated with a light film
of lubricating oil.
