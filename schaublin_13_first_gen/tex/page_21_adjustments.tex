\chapter{Adjustment}\label{chap:adjustments}

Only an experienced person should be entrusted with the operations of adjustment described below,
because they require the greatest of skill.

\section*{Milling Spindle Head}

\subsection*{Spindle Bearing}

The front-bearing is composed of a double row cylinder roller bearing 45, type SKF MN 3010 KX/SP

The back-bearing has two taper roller bearings 46, type 100.035 / 100.072 (accuray 2 microns) made
by La Précision Industrielle, Rueil-Malmaison (France)

These bearings are adjusted on mounting the milling machine. No readjustment is necessary until after
a long time of run.

\subsection*{Dismantling of the Spindle}

Previously the milling spindle head has to be taken off, taking into consideration the following
instructions

\begin{enumerate}
    \item Unscrew totally the set screw 57 (see page 25) and pull out the taper packing strip
    \item Remove the strip 135 after having loosened 3 screws
    \item Remove the milling spindle head towards the table, raising said head as much as possible in
\end{enumerate}

order to avoid a damage of the packing by the gear wheel.
The dismantling of the spindle is done as follows:

\begin{enumerate}
    \item Remove the coverplate 137 held by six screws
    \item Unscrew the locking screw 138
    \item Loosen screw 47 and unscrew entirely cover 48
    \item Take off nut 139 held by the safety washer on the spindle
    \item Push out carefully the spindle, striking its back-end slightly with a lead hammer
\end{enumerate}

\subsection*{Taking Up Radial Play within the front bearing}

\begin{enumerate}
    \item Determine the exact amount of radial play by means of a dial having a division of 0,001. (.00004 in.)
    \item Remove the spindle 26 (see "Dismantling the spindle")
    \item Tighten the nut 149 according to the radial play to be adjusted. A safety washer connects the ringnut 149 with the spindle.

    The slight taper of the inner race of bearing 45 obstructs a regular advance of the nut 140.
    To obtain the desired result, the nut should be struck concentrically by means of a tube placed
    on the spindle to cause a slight displacement of the inner race of the bearing 45 on the taper
    of the spindle 25, and nut 140 then retightened. By repeating this operation several times it will
    be found possible to turn nut 140 through the desired angle. Keep a careful check on the advance
    of mut 140, as it is difficult to restore the inner race of the bearing 45 if it is once moved
    too far forward on the taper.

    The advance of nut 140 is calculated, in respect of the radial play which has to be adjusted,
    as follows:

    \ul{Advance of nut 140 = bearing play to be adjusted in mm x 14}
    Pitch of nut 140 = 1,5 mm.

    \ul{Example:} A radial play of 0,01 mm has to be adjusted.
    Advance of nut 140 = 0,01 mm x 14 = 0,14 mm or a rotation of 36° 36' corresponding to a length
    of 20,5 mm on the full diameter of nut 140 (diam. 70 mm).

    \item Lock the nut 140 by means of the safety washer
    \item Replace spindle 26 and test again the radial play of the front-bearing; it should be 0,002 mm
    (.00008") in order to obtain faultless running conditions. This test shall be made with ball
    bearings 46 in position and with roller bearing 45 absolutely dry.
\end{enumerate}

\subsection*{Taking up radial and axial plain in the rear-bearing}

\begin{enumerate}
    \item Loosen screw 47
    \item Screw in the cover 48 (pitch 1,25 mm) according to the amount of play which has to be adjusted
    \item Firmly tighten screw 47 and test again the axial play of the back-bearing; it should be 0,01 mm
    (.0004") to obtain faultless running conditions. This test shall be done with taper roller bearings
    46 absolutely dry.
\end{enumerate}
