\chapter*{NOTE on the Maintenance of Coolant Pumps}

The coolant pump should be thoroughly cleaned for single-shift operation at least 2 times a year and for multi-shift operation at least 3 times a year.
For this purpose, it must be completely disassembled. All parts should be carefully cleaned in petroleum or benzine.
Tanks, pipes, grids, and filters should also be cleaned thoroughly.

Neglecting maintenance leads to rapid wear of the pump, especially when using so-called "soluble" oil emulsions.

The pump gland packings should be adjusted correctly or replaced if necessary.
For work without coolant fluid, the pump should be taken out of service.
Running the pump without liquid quickly causes serious defects.
If, even after observing the rules above, the pump continues to cause problems, it should be attributed to:

\begin{enumerate}
    \item The lubricant, especially the emulsion, is not changed frequently enough.
    Due to decomposition, the fatty part of the emulsion, together with chips (especially light metal chips) and other impurities,
    forms a sticky mass that clogs grids, filters, pipes, and fittings.
    \item The tanks are not thoroughly cleaned when changing the coolant fluid, so the new mixture becomes unusable immediately.
    \item Often, due to forgetfulness or negligence, the pump is not stopped, even when there is no suction of liquid.
    Due to its silent operation, it runs during breaks and often 24 hours a day.
\end{enumerate}

\textbf{The electric pump is an important and necessary accessory for all machine tools.}

It is used to bring the coolant fluid to the point where chip removal occurs to remove the heat generated,
lubricate, protect the machined surface from rust and corrosion, and also evacuate the chips during machining.

This process increases both cutting efficiency and tool life.
On the actual work, a good supply of coolant fluid influences precision and finishing quality.
